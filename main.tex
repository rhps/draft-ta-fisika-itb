\documentclass[12pt,a4paper]{report}
\renewcommand{\baselinestretch}{1.5}			%Mengubah spasi menjadi 1.5
\renewcommand{\chaptername}{\large{BAB }}		%Mengubah kata menjadi Bab
\renewcommand{\figurename}{\textbf{Gambar}}		%Mengubah kata menjadi Gambar
\renewcommand{\contentsname}{DAFTAR ISI}		%Mengubah kata menjadi Daftar Isi
\renewcommand{\listfigurename}{DAFTAR GAMBAR}	%Mengubah kata menjadi Daftar Gambar
\renewcommand{\listtablename}{DAFTAR TABEL}		%Mengubah kata menjadi Daftar Tabel
\renewcommand{\tablename}{\textbf{Tabel}}		%Mengubah kata menjadi Tabel
\renewcommand\bibname{DAFTAR PUSTAKA}			%Mengubah kata menjadi Referensi
\usepackage[left=4cm,top=3cm,right=3cm,bottom=3cm]{geometry}
\usepackage[font = small]{caption}
\usepackage{graphicx}			%Library untuk gambar
\usepackage{fancyhdr}			%Library untuk menggunakan header
\usepackage{caption}			%Library untuk mengubah caption
%\usepackage{subcaption}			%Library untuk mengubah subcaption
\usepackage{subfig}				%Library untuk membuat subfigure (gambar yang berseblahan dan berurutan)
\usepackage{tocbibind}			%Library untuk menambahkan daftar isi, gambar, dan tabel ke table of contents
\usepackage{parskip}			%Library untuk menghapus paragraf indent
\setlength{\parskip}{1em}		%Add space between paragraph
\usepackage{enumitem}			%Library edit enumerate dan itemize
\usepackage{titlesec}			%Library edit title spacing
\usepackage{sectsty}
\usepackage{multirow}
\usepackage{array}
\usepackage{booktabs}
\usepackage{mathtools}
\usepackage{amssymb, amsmath} 	% needed for math
\usepackage{siunitx}
\usepackage{url}
\usepackage{hyperref}
\usepackage[backend=bibtex,citestyle=authoryear]{biblatex}
\bibliography{reference}
%\bibliographystyle{ieeetr}		%Jenis referensi
%================Mengubah ukuran font pada section, subsection, dll================
\sectionfont{\normalsize}
\subsectionfont{\normalsize}
\titleformat{\section}{\large\bfseries}{\thesection}{1em}{}
\titlespacing\section{0pt}{12pt plus 4pt minus 2pt}{0pt plus 2pt minus 2pt}
\titlespacing\subsection{0pt}{12pt plus 4pt minus 2pt}{0pt plus 2pt minus 2pt}
\titlespacing\subsubsection{0pt}{12pt plus 4pt minus 2pt}{0pt plus 2pt minus 2pt}
%========================Mengubah spacing pada awal chapter========================
\titleformat{\chapter}[display]
	{\normalfont\large\bfseries\centering}{\chaptertitlename \thechapter \centering}{-10pt}{\large}
\titlespacing*{\chapter}{0pt}{-30pt}{20pt}
%==================================================================================

\begin{document}
\begin{titlepage}
	\centering
	{\textbf{RANCANG BANGUN SISTEM PENDETEKSI GAS ETILEN UNTUK MENENTUKAN TINGKAT KEMATANGAN BUAH MANGGA}}
	
	\vspace{2cm}
	\textbf{Tugas Akhir} \\
	Diajukan sebagai salah satu syarat untuk memperoleh gelar Sarjana Fisika dari Institut Teknologi Bandung
	
	\vspace{1cm}
	\textbf{Oleh: \\
	Moch Adha Agary Andi Paso \\
	10213040}
	
	\vspace{1cm}
	\begin{figure}[!htbp]
		\centering
		\includegraphics[width=3.5cm]{resources/itb_logo.png}
		\label{cover}
	\end{figure}
		
	\vspace{2cm}
	{\textbf{PROGRAM STUDI FISIKA \\
	FAKULTAS MATEMATIKA \& ILMU PENGETAHUAN ALAM \\
	INSTITUT TEKNOLOGI BANDUNG \\
	2017 \\}}
\end{titlepage}

\setcounter{page}{1}
\pagenumbering{roman}
\chapter*{\centering ABSTRAK}

\addcontentsline{toc}{chapter}{ABSTRAK}

\chapter*{\centering ABSTRACT}

\addcontentsline{toc}{chapter}{ABSTRACT}

\chapter*{\centering LEMBAR PENGESAHAN}

\chapter*{\centering PEDOMAN PENGGUNAAN}

\chapter*{\centering KATA PENGANTAR}

\addcontentsline{toc}{chapter}{KATA PENGANTAR}

\normalsize {\tableofcontents}
\normalsize {\listoffigures}
\normalsize {\listoftables}

\chapter{PENDAHULUAN}
\setcounter{page}{8}				%Mulai page numbering pada halaman 8
\pagenumbering{arabic}
	\section{Latar Belakang}
		
	\section{Rumusan Masalah}
		
	\section{Tujuan Penelitian}
		
	\section{Batasan Masalah}
		
	\section{Manfaat Penelitian}
	
	\section{Sistematika Penulisan}
	
\chapter{KAJIAN PUSTAKA}
	\section{Buah}
	
	\section{Etilen}
	
	\section{\textit{Infrared Radiation}}
	%Format Tabel
	\begin{table}[!htbp]
		\small
		\centering
		\caption{Pembagian Jenis Radiasi Infrared.}
		\label{tabel1}
		\begin{tabular}{|p{5.0cm}|p{2cm}|p{3.5cm}|}
		\hline
		\textbf{Description }	& \textbf{CIE }	& \textbf{Wavelength ($\mu m$)} \\
		\hline
		Near-Infrared				& IR-A	& 0.7  - 1.4 $\mu m$\\
		Near-Infrared				& IR-B	& 1.4  - 3 $\mu m$ \\
		Mid-wavelength Infrared		& IR-C	& 3 - 8 $\mu m$ \\
		Long-wavelength Infrared	& IR-C	& 8 - 15 $\mu m$ \\
		Far Infrared				& IR-C	& 15 - 1000 $\mu m$ \\
		\hline
		\end{tabular}
	\end{table}
	\vspace{-0.5cm}
	\subsection{Infrared Spectroscopy}
			
	\subsection{Frekuensi Vibrasi Molekul}
	%Format persamaaan
	\begin{equation}
		F = ma
		\label{newton}
	\end{equation}\\
		
	\section{Reflektansi, Absorbansi, dan Transmitansi}
		
	\subsection{Reflektansi}
		
	\subsection{Absorbansi}
		
	\subsection{Transmitansi}
		
	\section{\textit{Thermal Radiation Detector}}
		
\chapter{RANCANGAN SISTEM DAN EKSPERIMEN}
	\section{Desain Secara Umum}
			
	\section{Skema Deteksi Gas}
		
	\section{Desain Perangkat Keras}
	Bagian ini menjelaskan mengenai implementasi rangkaian secara terintegrasi mulai dari interface sensor thermopile sampai amplifier yang digunakan.

	Perujukan literatur dapat dilakukan dengan menambahkan entri baru di berkas. Tulisan ini merujuk pada \parencite{knuth2001art}
		
	\subsection{Blok Sensor}
		
	\subsection{Blok Amplifier}
			
	\subsection{Blok Source}
			
	\subsection{Blok Power}
			
	\section{Rancangan Eksperimen}
		
	\subsection{Penentuan Tegangan Optimum Kipas}
		
	\subsection{Penentuan Tegangan Referensi}
			
\chapter{HASIL DAN PEMBAHASAN}
	\section{Hasil}
	\subsection{Pengukuran Tegangan pada Kipas terhadap Sinyal yang Dihasilkan}
		
	\subsection{Pengukuran Pengaruh Berbagai Kematangan Buah terhadap Sinyal yang Dihasilkan}
		
	\section{Pembahasan}
	\subsection{Penentuan Tegangan Optimum}
		
	\subsection{Penentuan Tegangan Referensi}
		
\chapter{KESIMPULAN DAN SARAN}
	\section{Kesimpulan}
	\section{Saran}

\printbibliography
\chapter*{\centering LAMPIRAN}
\addcontentsline{toc}{chapter}{LAMPIRAN}

\end{document}